\documentclass[a4paper]{article}
\usepackage{amsmath}
\usepackage{amsfonts}
\usepackage{amssymb}

\usepackage{algorithm}
\usepackage[noend]{algpseudocode}

\usepackage{graphicx}
\usepackage{amsmath}
\usepackage[T2A]{fontenc}
\usepackage[utf8]{inputenc}
\usepackage[russian]{babel}
\usepackage{indentfirst}
\usepackage{amssymb}
\usepackage{graphicx}
\usepackage{fancyvrb}
\usepackage{listings}
\usepackage{tikz}
\usepackage{hhline}
\lstset{basicstyle=\ttfamily,
    escapeinside={||},
    mathescape=true}
%\usepackage{pdf}

\newcommand{\ds}{\displaystyle}
\newcommand{\al}{\alpha}
\newcommand{\be}{\beta}
\newcommand{\ga}{\gamma}
\newcommand{\z}{\par\addvspace{\medskipamount}
    \addtocounter{totaltasks}{1}%
    \textbf{\arabic{totaltasks}.}
}
\newcommand{\zD}{\par\addvspace{\medskipamount}
    \addtocounter{totaltasksD}{1}%
    \textbf{Д-\arabic{totaltasksD}.}
}
\newcommand{\za}{
    \textbf{a)}
}
\newcommand{\zb}{
    \textbf{b)}
}
\newcommand{\zc}{
    \textbf{c)}
}
\newcommand{\zd}{
    \textbf{d)}
}

\newcommand{\ra}{\rightarrow}
\newcommand{\Ra}{\Rightarrow}
\newcommand{\Lra}{\Leftrightarrow}
\newcommand{\cd}{\cdot}
\newcommand{\gs}{\geqslant}
\newcommand{\ls}{\leqslant}
\newcommand{\tx}{\text}
\newcommand{\ti}{\times}
\newcommand{\ov}{\overline}
\newcommand{\df}{\ds\frac}
\newcommand{\an}{\angle}
\newcommand{\tr}{\triangle}
\newcommand{\pa}{\parallel}
\newcommand{\ig}{\includegraphics}
\newcommand{\eq}{\equiv}
\renewcommand{\mod}{\operatorname{mod}}
\newcommand{\ba}{\begin{array}{c}}
    \newcommand{\ea}{\end{array}}
\renewcommand{\sb}{\left\{\ba}
\newcommand{\se}{\ea\right.}
\newcommand{\ovr}{\overrightarrow}
\newcommand{\om}{\omega}
\renewcommand{\tg}{\operatorname{tg}}
\renewcommand{\ctg}{\operatorname{ctg}}
\renewcommand{\phi}{\varphi}
\newcommand{\eps}{\varepsilon}
\newcommand{\Z}{\mathbb{Z}}
\newcommand{\N}{\mathbb{N}}
\newcommand{\Q}{\mathbb{Q}}
\newcommand{\F}{\mathbb{F}}
\newcommand{\R}{\mathbb{R}}
\newcommand{\de}{\mathop{\raisebox{-2pt}{\vdots}}}
\newcommand{\ub}{\underbrace}
\newcommand{\arc}{\smile\!}

\usepackage{wrapfig}
\usepackage[paperheight=20.5cm,paperwidth=14cm, left=0.7cm,right=0.7cm,top=1cm,bottom=1.5cm]{geometry}

\newcounter{totaltasks}

\begin{document}

\begin{center}
\Large\textbf{ТЗ семестрового проекта по информатике}
\end{center}

Состав команды : Жогов Александр, Панкратов Виктор, Сысак Михаил.

Тема : игра жанра RTS (Real Time Strategy).

Клиент состоит из $3$-x основных частей: интерфейс, ядро, ИИ.

Интерфейс включает в себя работу с вводом-выводом и рисовку происходящего, используя библиотеку SFML и общается с ядром посредством специальных методов.

Ядро можно тоже разделить на две части: внутренняя, которая занимается непосредственным изменением координат и состояний объектов, и внешняя (логическая), которая общается непосредственно с интерфейсом и служит для перевода абстрактных приказов для в более узкий класс и передает уже их внутреннему ядру.

ИИ --- демон, управляющий компьютерными игроками посредством общения с внешним ядром теми же методами, что и интерфейс.

Также прилагается сетевой итнерфейс для общения с ядром на удаленной машине.

Игровой процесс будет состоять из элементов жанра RTS. В игре основной будет тематика продуктов. Например раса печенек, сэндвичей, сыра и другого. Причем помимо внешнего вида у каждой будет по-своему уникальные здания, юниты и взаимодействие с противником, например раса сэндвичей сможет собирать себя из останков различных противников, и т.~д.




 

\end{document}